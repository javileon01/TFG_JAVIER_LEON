% Contenidos del capítulo.
% Las secciones presentadas son orientativas y no representan
% necesariamente la organización que debe tener este capítulo.

\section{Diagrama de casos de uso}

Identificación de actores y diagramas de casos de uso asociados a los requisitos funcionales.

\section{Diagrama de secuencia}
Para los casos de uso más importantes y complejos se deben especificar los diagramas de secuencia que describan el comportamiento de los objetos que intervienen en el caso de uso.
\section{Diagrama de clases de primer nivel}

Al final del análisis se debe presentar un diagrama de clases de primer nivel que muestre las clases principales del sistema y sus relaciones.
Para cada una de las clases que tengan un estado complejo o relevante, se debe especificar el diagrama de estados asociado o el diagrama de actividades en el que se involucra la clase.