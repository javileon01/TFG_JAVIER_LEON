% Contenidos del capítulo.
% Las secciones presentadas son orientativas y no representan
% necesariamente la organización que debe tener este capítulo.

% Introducción
En el presente capítulo se lleva a cabo un análisis del estado del arte con el fin de situar el 
proyecto en el contexto de soluciones existentes y tecnologías actuales. La plataforma 
web desarrollada en este trabajo tiene como objetivo facilitar el emparejamiento entre 
profesionales y proyectos mediante un sistema basado en competencias jerarquizadas. 
Por ello, resulta esencial estudiar qué herramientas similares existen ya en el mercado 
y cómo se enfrentan a problemas de emparejamiento, recomendación o gestión de 
talento y proyectos.

Primero se presentarán aplicaciones con funcionalidades relacionadas, analizando sus 
características principales, puntos en común con esta propuesta y diferencias clave. 
Este estudio permitirá identificar tanto aspectos que se consideran imprescindibles en 
una plataforma de este tipo, como oportunidades de mejora o innovación.

Posteriormente se realizará un análisis crítico de las tecnologías más relevantes para el 
desarrollo del sistema, justificando la elección final en cada caso a partir de criterios 
como la escalabilidad, facilidad de desarrollo, mantenimiento o compatibilidad entre 
componentes.

Por último, se listarán aquellas herramientas de soporte utilizadas durante el desarrollo 
del trabajo, como entornos de desarrollo, sistemas de control de versiones o editores 
de diagramas, las cuales han sido fundamentales para llevar a cabo el proyecto.


\section{Análisis de aplicaciones similares}
% Qué aplicaciones similares hay y en qué se diferencia de ellas la propuesta
En esta sección se analizarán distintas aplicaciones existentes que presentan elementos 
comunes con la \textbf{plataforma web desarrollada en este trabajo}, cuyo objetivo es 
facilitar el \textbf{emparejamiento entre profesionales y proyectos} en función de sus 
competencias. Para ello, se comentarán \textbf{soluciones reales} que abordan problemas 
similares, ya sea desde la perspectiva de \textbf{plataformas orientadas al empleo y el talento}, 
o desde el enfoque de \textbf{sistemas de emparejamiento inteligente basados en afinidad}.

Dado que el proyecto combina ideas presentes en entornos profesionales como LinkedIn \cite{linkedin}
y en algoritmos de emparejamiento como los utilizados por aplicaciones tipo Tinder \cite{tinder}, se 
ha optado por dividir este análisis en \textbf{dos bloques diferenciados}. En el primero se 
estudiarán \textbf{plataformas centradas en la gestión del talento y la búsqueda de empleo}, 
mientras que en el segundo se abordarán \textbf{sistemas cuyo núcleo es el emparejamiento 
basado en coincidencias}, con el objetivo de extraer ideas aplicables al 
\gls{recomendaciones} que se desea implementar.

Este análisis permitirá no solo \textbf{identificar funcionalidades clave y enfoques existentes}, 
sino también detectar \textbf{carencias o posibles áreas de mejora}, con el fin de proponer 
una solución más \textbf{adaptada, automatizada} y centrada en la 
\textbf{coincidencia de competencias concretas}.

\subsection{Plataformas web orientadas al empleo y la gestión del talento}

Dado que la plataforma web a desarrollar tiene como objetivo conectar profesionales con 
proyectos en función de sus competencias, las primeras aplicaciones a analizar son aquellas 
centradas en la búsqueda de empleo, la exposición del perfil profesional y la gestión del talento.

En esta sección se analizarán concretamente las plataformas \textbf{LinkedIn} \cite{linkedin}, 
\textbf{InfoJobs} \cite{infojobs} y \textbf{Yobalia} \cite{yobalia}, destacando sus funcionalidades principales y su grado 
de similitud con la solución propuesta en este trabajo.

\subsubsection{LinkedIn}
LinkedIn es una de las plataformas más conocidas y utilizadas a nivel mundial...


\subsection{Aplicaciones con sistemas de emparejamiento basados en coincidencias}


\section{Evaluación de tecnologías}\label{sec:evaluacion-tecnologias}
% Análisis crítico de las tecnologías y sistemas de despliegue posibles y por qué se han seleccionado unas concretas.
