% Contenidos del capítulo.
% Las secciones presentadas son orientativas y no representan
% necesariamente la organización que debe tener este capítulo.

% Introducción
En el presente capítulo se lleva a cabo un análisis del estado del arte con el fin de situar el 
proyecto en el contexto de soluciones existentes y tecnologías actuales. La plataforma 
web desarrollada en este trabajo tiene como objetivo facilitar el emparejamiento entre 
profesionales y proyectos mediante un sistema basado en competencias jerarquizadas. 
Por ello, resulta esencial estudiar qué herramientas similares existen ya en el mercado 
y cómo se enfrentan a problemas de emparejamiento, recomendación o gestión de 
talento y proyectos.

Primero se presentarán aplicaciones con funcionalidades relacionadas, analizando sus 
características principales, puntos en común con esta propuesta y diferencias clave. 
Este estudio permitirá identificar tanto aspectos que se consideran imprescindibles en 
una plataforma de este tipo, como oportunidades de mejora o innovación.

Posteriormente se realizará un análisis crítico de las tecnologías más relevantes para el 
desarrollo del sistema, justificando la elección final en cada caso a partir de criterios 
como la escalabilidad, facilidad de desarrollo, mantenimiento o compatibilidad entre 
componentes.

Por último, se listarán aquellas herramientas de soporte utilizadas durante el desarrollo 
del trabajo, como entornos de desarrollo, sistemas de control de versiones o editores 
de diagramas, las cuales han sido fundamentales para llevar a cabo el proyecto.


\section{Análisis de aplicaciones similares}
% Qué aplicaciones similares hay y en qué se diferencia de ellas la propuesta


\section{Evaluación de tecnologías}\label{sec:evaluacion-tecnologias}
% Análisis crítico de las tecnologías y sistemas de despliegue posibles y por qué se han seleccionado unas concretas.
