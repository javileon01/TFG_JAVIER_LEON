% Contenidos del capítulo.
% Las secciones presentadas son orientativas y no representan
% necesariamente la organización que debe tener este capítulo.

% Diagramas de clases, de secuencia, de despliegue, diseño de
% pantallas, etc

\section{Arquitectura de componentes}

A partir de la especificación realizada en la sección \ref{sec:especificaciones-sistema} del Capítulo \ref{ch:requisitos} y la tecnología seleccionada en la sección \ref{sec:evaluacion-tecnologias} del Capítulo \ref{ch:estado-arte}, se debe construir el diagrama de componentes UML que represente la arquitectura de referencia del sistema. 
En este diagrama se deben identificar los componentes principales del sistema y las relaciones entre ellos, pasando a detallar los subcomponentes de cada uno de ellos en secciones posteriores.

\section{Despliegue del sistema}

En esta sección se debe presentar el diagrama de despliegue del sistema, que muestre la arquitectura física del sistema y cómo se distribuyen los componentes en los nodos físicos o virtuales que lo componen.
En otras palabras, para cada componente principal del sistema se identifica en que servidor físico o virtual se va a ejecutar.

Se puede definir un despligue de desarrollo (el despliegue que se tiene en el portátil mientras se trabaja), uno de producción (despliegue final o real) y uno de pruebas (lo más parecido posible al real).

\section{Modelo de datos}

En este caso se puede especificar el diagrama de clases final del diseño atendiendo al lenguaje de programación y a la tecnología seleccionada.
Además será necesario, como mínimo, especificar el modelo físico de datos, en el lenguaje SQL propio de la base de datos relacional seleccionada, o el modelo de documentos, si se utiliza una base de datos NoSQL. 
